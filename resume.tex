%!TEX TS-program = xelatex
%!TEX encoding = UTF-8 Unicode
\documentclass[11pt, a4paper]{awesome-cv}

% Configuración de márgenes
\geometry{left=1.4cm, top=.8cm, right=1.4cm, bottom=1.8cm, footskip=.5cm}
\fontdir[fonts/]
\colorlet{awesome}{awesome-red}
\setbool{acvSectionColorHighlight}{true}
\renewcommand{\acvHeaderSocialSep}{\quad\textbar\quad}

%-------------------------------------------------------------------------------
%   REDEFINICIÓN DESTRUCTURAL DE CVENTRY (CORREGIDA)
%-------------------------------------------------------------------------------
\renewcommand*{\cventry}[5]{%
  \vspace{-2.0mm}
  \setlength\tabcolsep{0pt}
  \setlength{\extrarowheight}{0pt}
  % 1. ENCABEZADO (Mantenemos la tabla solo para esto, garantiza alineación perfecta de fechas)
  \begin{tabular*}{\textwidth}{@{\extracolsep{\fill}} L{\textwidth - 4.5cm} R{4.5cm}}
    \ifempty{#2#3}
      {\entrypositionstyle{#1} & \entrydatestyle{#4} \\}
      {\entrytitlestyle{#2} & \entrydatestyle{#4} \\
       \entrypositionstyle{#1} & \entrylocationstyle{#3} \\}
  \end{tabular*}%
  % 2. CUERPO (Bullets)
  % Rompemos el párrafo para salir de la tabla, quitamos indentación y aplicamos el ESTILO ORIGINAL
  \par\nobreak
  \vspace{-1mm} % Ajuste fino para pegar los bullets al título
  \noindent % <--- CRÍTICO: Evita que el texto se mueva a la derecha
  \descriptionstyle{#5} % <--- CRÍTICO: Restaura la fuente Roboto y el tamaño correcto
  \par\vspace{5mm} % Espacio seguro después de cada bloque
}

%-------------------------------------------------------------------------------
%	DATOS PERSONALES
%-------------------------------------------------------------------------------
\name{Tomás}{Pont Vergés}
\position{Data Science{\enskip\cdotp\enskip}Sociology{\enskip\cdotp\enskip}Communication}
\address{Valle Grande 1182, Vicente López, Argentina}
\mobile{(+54) 911 3015-8395}
\email{tomaspont@gmail.com}
\github{tepeve}
\linkedin{tepeve}
\extrainfo{\faBirthdayCake\ \ 16/05/1984}

\begin{document}

\makecvheader[C]
\makecvfooter{\today}{Tomás Pont Verges~~~·~~~Curriculum Vitae}{\thepage}

%-------------------------------------------------------------------------------
%	SECTION TITLE
%-------------------------------------------------------------------------------
\cvsection{About Me}


%-------------------------------------------------------------------------------
%	CONTENT
%-------------------------------------------------------------------------------
\begin{cvparagraph}

%---------------------------------------------------------
Communications Professional & Sociologist with solid background in media, press, and advertising, currently transitioned into Data Science and Engineering. Hands-on experience in Python, SQL, and data visualization, specializing in ETL processes and the deployment of analytical and predictive models for mass media campaigns, public management, and retail marketing. My focus is on collecting, cleaning, and structuring data to transform it into meaningful, high-value insights.
\end{cvparagraph}


%-------------------------------------------------------------------------------
%	SECTION TITLE
%-------------------------------------------------------------------------------
\cvsection{Skills}


%-------------------------------------------------------------------------------
%	CONTENT
%-------------------------------------------------------------------------------
\begin{cvskills}
%---------------------------------------------------------
  \cvskill
    {SQL} % Category
    {PostgreSQL, BigQuery, Redshift, Dbt, Dataform } % Skills    
%---------------------------------------------------------
  \cvskill
    {Python} % Category
    {Numpy, Pandas, Scikit-Learn, PYMC, LightFM, Geopandas} % Skills
%---------------------------------------------------------
  \cvskill
    {R} % Category
    {Tidyverse, Tidymodels, Sf} % Skills
%---------------------------------------------------------
  \cvskill
    {Visualización} % Category
    {Looker, Power BI, Py: Matplotlib, Seaborn, Plotly, Streamlit, R: Ggplot2, Leaflet, Shiny} % Skills
%---------------------------------------------------------
  \cvskill
    {Servicios y Herramientas OPS } % Category
    {GCP, GIT, Docker, Airflow, MLflow} % Skills   
%---------------------------------------------------------
  \cvskill
    {Otros lenguajes} % Category
    {Bash, PHP, LaTeX, HTTP, CSS} % Skills
%---------------------------------------------------------
  \cvskill
    {Idiomas} % Category
    {Español (Nativo), Inglés (Intermedio/B2)}  % Skills
%---------------------------------------------------------
\end{cvskills}


%-------------------------------------------------------------------------------
%	EXPERIENCIA
%-------------------------------------------------------------------------------
\cvsection{Professional Experience}

\begin{cventries}

% Arg 1: Cargo
% Arg 2: Empresa
% Arg 3: Ubicación/Modalidad
% Arg 4: Fecha
% Arg 5: Descripción (Bullets)

  \cventry%
    {Data Scientist}%
    {Andata Consultora (Client: Mondelēz International)}%
    {Full-time}%
    {Mar 2025 - Dec 2025}%
    {%
      \begin{cvitems}
        \item {Developed a hybrid recommendation system (XGBoost + Collaborative Filtering + Business Rules) for 180k points of sale, optimizing product conversion across the retail channel.}
        \item {Built Bayesian regression models to predict product elasticity and demand for supermarket chains based on different promotional scenarios.}
      \end{cvitems}%
    }

  \cventry%
    {Data Analyst}%
    {Gustavo Grobocopatel (Client: Consejo Federal de Inversiones)}%
    {Part-time}%
    {Aug 2024 - Nov 2024}%
    {%
      \begin{cvitems}
      \item {Collected, integrated, and analyzed data for the report "Catamarca: Notes for Sustainable Development," a diagnostic study used to formulate productive development policies for the province.}      \end{cvitems}%
    }

\cventry%
    {Journalism Producer}%
    {Ekeko Producciones (Client: ARD Radio - Germany)}%
    {Part-time}%
    {Jul 2024 - Jan 2026}%
    {%
      \begin{cvitems}
        \item {Produced reports and conducted interviews for the South American bureau of the German public broadcaster ARD-Radio, the largest public broadcasting network in the European Union.}
      \end{cvitems}%
    }
    
  \cventry%
    {Digital Content Production Coordinator / Data Analyst}%
    {Presidency of Argentina}%
    {Full-time}%
    {Dec 2021 - Dec 2023}%
    {%
      \begin{cvitems}
        \item {End-to-end management of large-scale campaign lifecycles (TV, Out-of-Home, Digital): from creative development to impact assessment and performance metrics.}
        \item {Processed and analyzed surveys for audience segmentation and the definition of strategic messaging.}
        \item {Designed and implemented a content repository and logging system: created metadata tagging taxonomies and developed Looker Studio dashboards for KPI monitoring and asset auditing.}
      \end{cvitems}%
    }

  \cventry%
    {Community Manager}%
    {Latin American Council of Social Sciences (CLACSO)}%
    {Part-time}%
    {Feb 2021 - Apr 2023}%
    {%
      \begin{cvitems}
        \item {Developed digital channel strategies and provided institutional communications support for the Executive Office.}
      \end{cvitems}%
    }

  \cventry%
    {Communications \& Press Advisor}%
    {PAMI - National Institute of Social Services}%
    {Full-time}%
    {Jun 2020 - Nov 2021}%
    {%
      \begin{cvitems}
        \item {Coordinated nationwide communications across 38 regional management units.}
        \item {Designed media plans, managed press relations, and provided media coaching for spokespersons.}
      \end{cvitems}%
    }

  \cventry%
    {Journalist and Producer}%
    {Radio Nacional Argentina}%
    {Fulltime}%
    {Jun 2009 - May 2020}%
    {%
      \begin{cvitems}
        \item {Coordinated research teams, managed news production (AM/FM), and authored web content.}      \end{cvitems}%
    }

  \cventry%
    {Journalist - Audiovisual Production}%
    {Craneo Films}%
    {Part-time}%
    {Feb 2018 - Oct 2019}%
    {%
      \begin{cvitems}
        \item {Managed the end-to-end production and scriptwriting for the documentary series "El Camino de las Letras" and "Los Trabajos del Futuro".}
      \end{cvitems}%
    }

  \cventry%
    {Research Assistant}%
    {Center for Legal and Social Studies (CELS)}%
    {Part-time}%
    {Jun 2016 - May 2017}%
    {%
      \begin{cvitems}
        \item {Conducted data analysis for public policy evaluations within the Economic, Social, and Cultural Rights department.}
      \end{cvitems}%
    }

  \cventry%
    {Front-end Developer}%
    {HRsmart}%
    {Full-time}%
    {Apr 2007 - Jun 2009}%
    {%
      \begin{cvitems}
        \item {Designed and developed HR management solutions using PHP, MySQL, HTML, and CSS.}
      \end{cvitems}%
    }

  \cventry%
    {Journalist}%
    {Grupo Octubre - Revista Caras y Caretas}%
    {Partime}%
    {Mar 2005 - Jun 2009}%
    {%
      \begin{cvitems}
        \item {Staff writer for the "Politics and Society" section.}      
      \end{cvitems}%
    }

\end{cventries}


%-------------------------------------------------------------------------------
%	SECTION TITLE
%-------------------------------------------------------------------------------
\cvsection{Educación}


%-------------------------------------------------------------------------------
%	CONTENT
%-------------------------------------------------------------------------------
\begin{cventries}

  \cventry
    {Cloud Data Engineering en AWS} % Degree
    {ITBA} % Institution
    {Sept. 2025 - Nov. 2026} % Date(s)
    {EN CURSO} % Location
    {}
    \vspace{-7mm} % <--- Juega con este número (-3mm, -5mm) para subir el siguiente item

  \cventry
    {Data Science en Python} % Degree
    {Digital House} % Institution
    {Oct. 2023 - May. 2024} % Date(s)
    {} % Location
    {}
    \vspace{-7mm} % <--- Juega con este número (-3mm, -5mm) para subir el siguiente item

  \cventry
    {Big Data e Inteligencia Territorial en R}
    {FLACSO (Facultad Latinoamericana de Ciencias Sociales)}
    {Abr. 2022 - Dic. 2022}
    {}
    {}
    \vspace{-7mm} % <--- Juega con este número (-3mm, -5mm) para subir el siguiente item

  \cventry
    {Ciencia de Datos para Políticas Públicas}
    {Universidad Torcuato Di Tella}
    {May. 2020 - Sept. 2020}
    {}
    {}
    \vspace{-7mm} % <--- Juega con este número (-3mm, -5mm) para subir el siguiente item

  \cventry
    {Licenciado en Sociología (Promedio: 7.88)}
    {Universidad de Buenos Aires (UBA)}
    {2010 - 2019}
    {}
    {}
    \vspace{-7mm} % <--- Juega con este número (-3mm, -5mm) para subir el siguiente item

\end{cventries}

\end{document}